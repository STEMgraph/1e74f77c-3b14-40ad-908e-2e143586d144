\learningobjective{At the end of this challenge, the scholar will be able to use `std::transform` to apply transformations to elements of a range in C++.}
\begin{challenge}
    \chatitle{Using `std::transform` to manipulate container elements}
    \begin{chadescription}
    The C++ standard algorithm `std::transform` is a powerful tool for applying a transformation function to elements in a range. 
    It modifies each element based on the provided operation and can store the transformed results in the same container or a different one.

    `std::transform` is highly versatile and can:
    \begin{itemize}
        \item Apply mathematical operations to numeric data.
        \item Convert data types or formats (e.g., strings to uppercase).
        \item Combine two ranges into one, element-wise.
    \end{itemize}

    In this challenge, we will:
    \begin{enumerate}
        \item Learn the syntax and basic usage of `std::transform`.
        \item Transform elements in a single container.
        \item Combine two ranges to produce a new range.
    \end{enumerate}
    \end{chadescription}

    \begin{task}
    Write a C++ program to double the values in a `std::vector<int>` using `std::transform`.
    \begin{enumerate}
        \item Include the `<algorithm>` and `<vector>` headers.
        \item Create a `std::vector<int>` with values `{1, 2, 3, 4, 5}`.
        \item Use `std::transform` with a lambda function to double each element.
        \item Store the results in the same container and print them.
    \end{enumerate}
    \begin{questions}
        \item What happens if you use `std::transform` with different input and output containers?
        \item Can the transformation function modify the input range directly?
    \end{questions}
    \end{task}

    \begin{advice}
        Remember, `std::transform` is non-destructive when the output iterator is different from the input iterator.
    \end{advice}

    \begin{task}
    Modify the program to convert a `std::vector<std::string>` of lowercase words to uppercase.
    \begin{enumerate}
        \item Create a `std::vector<std::string>` with values `{"hello", "world", "c++"}`.
        \item Use `std::transform` with a lambda function and the `<cctype>` function `std::toupper` to convert characters to uppercase.
        \item Print the transformed strings.
    \end{enumerate}
    \begin{questions}
        \item How does `std::transform` handle strings and their characters?
        \item Why is it necessary to use `std::toupper` with a cast for character transformations?
    \end{questions}
    \end{task}

    \begin{advice}
        Use `std::transform` with range-based operations for more readable and concise code.
    \end{advice}

    \begin{task}
    Use `std::transform` to compute the sum of two vectors element-wise.
    \begin{enumerate}
        \item Create two `std::vector<int>` objects with values `{1, 2, 3}` and `{4, 5, 6}`.
        \item Use `std::transform` with a lambda function to compute the sum of corresponding elements.
        \item Store the result in a third `std::vector<int>` and print it.
    \end{enumerate}
    \begin{questions}
        \item What happens if the two input vectors have different sizes?
        \item How does `std::transform` ensure element-wise processing when combining two ranges?
    \end{questions}
    \end{task}

    \begin{advice}
        Always ensure the output container has enough space to hold the results before using `std::transform` with separate input and output ranges.
    \end{advice}

\end{challenge}
